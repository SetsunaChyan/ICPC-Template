\documentclass[twocolumn,a4]{article}
\usepackage{xeCJK} % For Chinese characters
\usepackage{amsmath, amsthm}
\usepackage{listings,xcolor}
\usepackage{geometry} % 设置页边距
\usepackage{fontspec}
\usepackage{graphicx}
\usepackage{fancyhdr} % 自定义页眉页脚
\setsansfont{Consolas} % 设置英文字体
\setmonofont[Mapping={}]{Consolas} % 英文引号之类的正常显示,相当于设置英文字体
\geometry{left=1cm,right=1cm,top=2cm,bottom=0.5cm} % 页边距
\setlength{\columnsep}{30pt}
\newcommand{\addcpp}[1]{\input{cpp/#1}} 
% \setlength\columnseprule{0.4pt} % 分割线

%========================页眉、页脚、代码格式设置=======================%
% 页眉、页脚设置
\pagestyle{fancy}
% \lhead{CUMTB}
\lhead{\CJKfamily{hei} USST ACM-ICPC Template}
\chead{}
% \rhead{Page \thepage}
\rhead{\CJKfamily{hei} 第 \thepage 页}
\lfoot{} 
\cfoot{}
\rfoot{}
\renewcommand{\headrulewidth}{0.4pt} 
\renewcommand{\footrulewidth}{0.4pt}

% 代码格式设置
\lstset{
    language    = c++,
    numbers     = left,
    numberstyle = \tiny,
    breaklines  = true,
    captionpos  = b,
    tabsize     = 4,
    frame       = single,
    columns     = fullflexible,
    commentstyle = \color[RGB]{0,128,0},
	keywordstyle ={
		\color[RGB]{0,51,153}
		\fontspec{Consolas Bold}	
	},
    basicstyle   = \small\ttfamily,
    stringstyle  = \color[RGB]{148,0,209}\ttfamily,
	rulesepcolor = \color{red!20!green!20!blue!20},
    showstringspaces = false,                    
}
%========================页眉、页脚、代码格式设置=======================%

%===============================标题和目录==============================%
\title{\CJKfamily{hei} \bfseries 小小青蛙听风就是雨}
\author{HiedanoAkyuu、Oneman233、KR12138}
\renewcommand{\today}{\number\year 年 \number\month 月 \number\day 日}

\begin{document}\small
\begin{titlepage}
\maketitle
\end{titlepage}

\newpage
\pagestyle{empty}
\renewcommand{\contentsname}{目录}
\tableofcontents
\newpage\clearpage
\newpage
\pagestyle{fancy}
\setcounter{page}{1}   %new page
%===============================标题和目录==============================%

%================================正文部分===============================%
\section{字符串}
	\subsection{KMP}
		%\addcpp{}
	\subsection{EX-KMP}
		\addcpp{string/EX-KMP}
	\subsection{Manacher}
		%\addcpp{}
	\subsection{串的最小表示}
		%\addcpp{}
	\subsection{后缀数组}
		\subsubsection{倍增SA}
			%\addcpp{}
		\subsubsection{DC3}
			%\addcpp{}
	\subsection{回文自动机}
		%\addcpp{}
	\subsection{AC自动机	}
		\subsubsection{多模匹配}
			\addcpp{string/Aho-CorasickAutomaton}
		\subsubsection{自动机上DP}
			%\addcpp{}
	\subsection{后缀自动机}
		%\addcpp{}

\section{计算几何}
	\subsection{二维几何}
		\addcpp{geometry/geometry-2D}
	\subsection{三维几何}
		%\addcpp{}
	
\section{图论}
	\subsection{最短路}
		\subsubsection{Dijkstra}
			%\addcpp{}
		\subsubsection{SPFA}
			%\addcpp{}
		\subsubsection{Floyd}
			%\addcpp{}
		\subsubsection{负环}
			%\addcpp{}
		\subsubsection{差分约束}
			%\addcpp{}
	\subsection{最小生成树}
		\subsubsection{Prim}
			%\addcpp{}
		\subsubsection{Kruskal}
			%\addcpp{}
		\subsubsection{最小生成树计数}
			%\addcpp{}
		\subsubsection{次小生成树}
			%\addcpp{}
		\subsubsection{最小乘积生成树}
			%\addcpp{}
	\subsection{树的直径}
		%\addcpp{}
	\subsection{LCA}
		\subsubsection{Tarjan离线}
			%\addcpp{}
		\subsubsection{倍增LCA}
			%\addcpp{}
	\subsection{无向图与有向图联通性}
		\subsubsection{割点}
			%\addcpp{}
		\subsubsection{桥}
			%\addcpp{}
		\subsubsection{e-DCC}
			%\addcpp{}
		\subsubsection{v-DCC}
			%\addcpp{}
		\subsubsection{SCC}
			%\addcpp{}
		\subsubsection{2-SAT}
			%\addcpp{}
		\subsubsection{支配树}
			%\addcpp{}
	\subsection{二分图}
		\subsubsection{最大匹配-匈牙利}		
			%\addcpp{}
		\subsubsection{带权匹配-KM}
			%\addcpp{}
		%\subsubsection{最小点覆盖}
		%\subsubsection{最大独立集}
		%\subsubsection{DAG最小路径点覆盖}
	\subsection{网络流}
		\subsubsection{最大流-Dinic}
			\addcpp{graph/MF_dinic}
		\subsubsection{最小费用最大流-Dij+Dinic}
			\addcpp{graph/MSMF_dij_dinic}
		%\subsubsection{最小费用最大流-SPFA+Dinic}
			%\addcpp{}
		\subsubsection{上下界流}
			%\addcpp{}
		%\addcpp{}
	\subsection{欧拉路}
		%\addcpp{}
	\subsection{Prufer序列}
		%\addcpp{}

\section{数据结构}
	\subsection{树状数组}
		%\addcpp{}
	\subsection{线段树}
		\subsubsection{多操作线段树}
			%\addcpp{}
		\subsubsection{吉司机线段树}
			\addcpp{datastruct/minSeg}
		\subsubsection{扫描线}
			%\addcpp{}
	\subsection{RMQ}
		\subsubsection{一维}
			%\addcpp{}
		\subsubsection{两维}
			%\addcpp{}
	\subsection{树链剖分}
		\subsubsection{点剖分}
			\addcpp{datastruct/heavy_light_decomposition}
		\subsubsection{边剖分}
			%\addcpp{}
	\subsection{平衡树}
		\subsubsection{Treap}
			\addcpp{datastruct/treap}
		\subsubsection{Splay}
			%\addcpp{}
	\subsection{动态树}
		%\addcpp{}
	\subsection{主席树}
		%\addcpp{}
	\subsection{树套树}
		\subsubsection{线段树套Treap}
			\addcpp{datastruct/seg_treap}
		\subsubsection{树状数组套线段树}
			\addcpp{datastruct/fenwick_seg}
	\subsection{K-D Tree}
		%\addcpp{}
	\subsection{分治}
		\subsubsection{CDQ}
			%\addcpp{}
		\subsubsection{点分治}
			%\addcpp{}
		\subsubsection{dsu on tree}
			%\addcpp{}
		\subsubsection{整体二分}
			%\addcpp{}
	\subsection{分块}
		\subsubsection{普通分块}
			%\addcpp{}
		\subsubsection{莫队}
			%\addcpp{}
	\subsection{线性基}
		%\addcpp{}
	\subsection{珂朵莉树}
		%\addcpp{}
	\subsection{跳舞链}
		%\addcpp{}

\section{动态规划}
	\subsection{SOS}
		%\addcpp{}
	\subsection{动态DP}
		%\addcpp{}
	\subsection{插头DP}
		%\addcpp{}

\section{数学}
	\subsection{矩阵类}
		%\addcpp{}
	\subsection{质数筛}
		\subsubsection{埃筛}
			%\addcpp{}
		\subsubsection{线筛}
			%\addcpp{}
	\subsection{质数判定}
		\subsubsection{Miller Rabin}
			%\addcpp{}
	\subsection{质因数分解}
		\subsubsection{Pollard-Rho}
			%\addcpp{}
	\subsection{逆元}
		\subsubsection{EX-GCD求逆元}
			%\addcpp{}
		\subsubsection{线性筛逆元}
			%\addcpp{}
		\subsubsection{阶乘逆元}
			%\addcpp{}
	\subsection{欧拉函数}
		\subsubsection{欧拉线筛}
			%\addcpp{}
		\subsubsection{求单个数的欧拉函数}
			%\addcpp{}
		\subsubsection{欧拉降幂}
			%\addcpp{}
		\subsubsection{一般积性函数求法}
			%\addcpp{}
	\subsection{EX-GCD}
		%\addcpp{}
	\subsection{CRT}
		%\addcpp{}	
	\subsection{N次剩余}
		%\addcpp{}
	\subsection{数论分块}
		%\addcpp{}
	\subsection{高斯消元}
		\subsubsection{普通消元}
			%\addcpp{}
		\subsubsection{异或方程组消元}
			%\addcpp{}
	\subsection{莫比乌斯反演}
		\subsubsection{莫比乌斯函数}
			%\addcpp{}
		\subsubsection{杜教筛}
			%\addcpp{}
		\subsubsection{洲阁筛}
			%\addcpp{}
		\subsubsection{min25筛}
			%\addcpp{}
	\subsection{BSGS}
		%\addcpp{}
	\subsection{FFT}
		%\addcpp{}
	\subsection{FWT}
		%\addcpp{}
	\subsection{NTT}
		%\addcpp{}
	\subsection{数值计算}
		\subsubsection{辛普森}
			%\addcpp{}
		\subsubsection{自适应辛普森}
			%\addcpp{}
	\subsection{康拓展开}
		%\addcpp{}
	\subsection{卢卡斯定理}
		%\addcpp{}

\section{其他}
	\subsection{快读快写}
		%\addcpp{}
	\subsection{约瑟夫环}
		%\addcpp{}
	\subsection{悬线法}
		%\addcpp{}
	\subsection{蔡勒公式}
		%\addcpp{}
	\subsection{三角公式}
		%\addcpp{}
	\subsection{海伦公式}
		%\addcpp{}
	\subsection{匹克定理}
		%\addcpp{}	
	\subsection{组合计数}
		\subsubsection{计数原理}
			%\addcpp{}
		\subsubsection{卡特兰数}
			%\addcpp{}
		\subsubsection{Polya}
			%\addcpp{}
		\subsubsection{二项式反演公式}
			%\addcpp{}
		\subsubsection{斯特林反演公式}
			%\addcpp{}
		\subsubsection{组合数恒等式}
			%\addcpp{}
			
%==============================正文部分==============================%
\end{document}