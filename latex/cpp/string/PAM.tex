\begin{lstlisting}
#include <bits/stdc++.h>
//jisuanke 41389

/*
	fail[x]:x节点失配之后跳转到不等于自身的最长后缀回文子串
	len[x]:以x结尾的最长回文子串长度
	diff[x]: 与”以x结尾的最长回文子串“本质不同的子串个数
	same[x]:与”以x结尾的最长回文子串“本质相同的子串个数
	(注意上面两个完全相反)
	son[x][c]:编号为x的节点表示的回文子串在两边添加字符c之后变成的回文子串编号
	s[x]:第x次添加的字符,s数组即原字符串
	tot:总节点个数,节点编号由0到tot-1
	last:最后一个新建立节点的编号
	cur:当前节点在PAM上的父亲编号
*/

#define int long long
using namespace std;
const int N=1e6+5;

struct PAM
{
	int tot,last,n,cur;
	int fail[N],len[N],same[N],diff[N],son[N][26];
	char s[N];
	int get(int p,int x)
	{
		while(s[x-len[p]-1]!=s[x])
			p=fail[p];
		return p;
	}
	int newnode(int x)
	{
		len[tot]=x;
		return tot++;
	}
	void build()
	{
		scanf("%s",s+1);
		s[0]=-1,fail[0]=1,last=0;
		newnode(0),newnode(-1);
		for(n=1;s[n];++n)
		{
			s[n]-='a';
			cur=get(last,n);
			if(!son[cur][s[n]])
			{
				int now=newnode(len[cur]+2);
				fail[now]=son[get(fail[cur],n)][s[n]];
				diff[now]=diff[fail[diff[now]]]+1;
				son[cur][s[n]]=now;
			}
			same[last=son[cur][s[n]]]++;
		}
		for(int i=tot-1;i>=0;--i)
			same[fail[i]]+=same[i];
	}
}pam;

int v[26],ans=0;
void dfs(int x,int now)
{
	if(pam.len[x]>0) ans+=pam.same[x]*now;
	for(int i=0;i<26;++i)
	{
		if(pam.son[x][i]!=0)
		{
			if(!v[i])
			{
				v[i]=1;
				dfs(pam.son[x][i],now+1);
				v[i]=0;
			}
			else dfs(pam.son[x][i],now);
		}
	}
}

signed main()
{
	pam.build();
	dfs(0,0);//even string
	dfs(1,0);//odd string
	printf("%lld",ans);
	return 0;
}
\end{lstlisting}