\begin{lstlisting}
#include <bits/stdc++.h>
#define int long long
//luogu P3469

/*
	tarjan求割点的算法中,如果不保证连通性,应该使用被注释掉的遍历方法
	part数组储存了被这个割点分成的不同的几块各自的大小
*/

using namespace std;
const int N=100005;

int n,m,x,y;
vector<int> e[N],part[N];
bool is[N];
int dfn[N],low[N],timer=0;
int sz[N];

void tarjan(int u,int f)
{
	dfn[u]=low[u]=++timer;
	sz[u]++;//
	int son=0,tmp=0;
	for(auto v:e[u])
	{
		if(dfn[v]==0)
		{
			tarjan(v,u);
			sz[u]+=sz[v];//
			low[u]=min(low[u],low[v]);
			if(low[v]>=dfn[u]&&u!=f)
			{
				is[u]=1;
				tmp+=sz[v];//
				part[u].push_back(sz[v]);//
			}
			if(u==f) son++;
		}
		low[u]=min(low[u],dfn[v]);
	}
	if(son>=2&&u==f) is[u]=1;//point on the top
	if(is[u]&&n-tmp-1!=0)
		part[u].push_back(n-tmp-1);//
}

signed main()
{
	scanf("%lld%lld",&n,&m);
	for(int i=1;i<=m;++i)
	{
		scanf("%lld%lld",&x,&y);
		e[x].push_back(y),e[y].push_back(x);
	}
	/*
	for(int i=1;i<=n;++i)
		if(!dfn[i]) tarjan(i,i);
	*/
	tarjan(1,0);
	for(int i=1;i<=n;++i)
	{
		if(!is[i]) printf("%lld\n",2*(n-1));
		else{
			int tmp=0;
			for(auto j:part[i])
				tmp+=j*(j-1);
			printf("%lld\n",n*(n-1)-tmp);
		}
	}
	return 0;
}
\end{lstlisting}