\begin{lstlisting}
#include <bits/stdc++.h>
using namespace std;
//luogu P4782

/*
	2-SAT用于求解有n个布尔变量x1-xn和m个需要满足的条件
	每个条件形式为xi=0(1)||xj=0(1),是否有可行解
	注意要开两倍空间建反向边
	2−SAT输出方案的方法为:先把缩点之后的新图进行拓扑排序,然后判断每个点i,如果i所在强连通分量的拓扑序在i'所在的强连通分量的拓扑序之后,那么第i场游戏使用该地图适合的第一种赛车,否则使用第二种赛车。
	但是由于Tarjan求强连通分量就是按拓扑排序的逆序给出的,所以直接使用强连通分量编号判断即可。
	即如果bel[]为每个点的所在强连通分量编号,那么判断为:如果bel[i]<bel[i'],那么使用该地图适合的第一种赛车,否则使用第二种赛车。
*/

const int N=2e6+5;

int n,m,a,va,b,vb;
int dfn[N],low[N],timer=0;
stack<int> s;
bool vis[N];
vector<int> e[N];
int co[N],color=0;

void add(int x,int y)
{
	e[x].push_back(y);
}

void tarjan(int u)
{
	dfn[u]=low[u]=++timer;
	s.push(u);
	vis[u]=1;
	for(auto v:e[u])
	{
		if(!dfn[v])
			tarjan(v),
			low[u]=min(low[u],low[v]);
		else if(vis[v])
			low[u]=min(low[u],dfn[v]);
	}
	if(low[u]==dfn[u])
	{
		int v;
		color++;
		do
		{
			v=s.top();
			s.pop();
			vis[v]=0;
			co[v]=color;
		}
		while(u!=v);
	}
}

bool solve()
{
	for(int i=1;i<=2*n;++i)
		if(!dfn[i]) tarjan(i);
	for(int i=1;i<=n;++i)
		if(co[i]==co[i+n])
			return 0;
	return 1;
}

int main()
{
	scanf("%d%d",&n,&m);
	for(int i=1;i<=m;++i)
	{
		scanf("%d%d%d%d",&a,&va,&b,&vb);
		int nota=va^1,notb=vb^1;
		add(a+nota*n,b+vb*n);//not a and b
		add(b+notb*n,a+va*n);//not b and a
	}
	if(solve())
	{
		puts("POSSIBLE");
		for(int i=1;i<=n;++i)
			printf("%d ",co[i]>co[i+n]);
	}
	else puts("IMPOSSIBLE");
	return 0;
}
\end{lstlisting}