\begin{lstlisting}
#include <bits/stdc++.h>
//hdu 2255
using namespace std;

/*
	KM仅用于最大带权匹配一定是最大匹配的情况中
*/

const int N=305;
const int inf=0x3f3f3f3f;

int n,mp[N][N];
int la[N],lb[N],delta;
bool va[N],vb[N];
int match[N];

bool dfs(int x)
{
	va[x]=1;
	for(int y=1;y<=n;++y)
	{
		if(!vb[y]){
			if(la[x]+lb[y]==mp[x][y])
			{
				vb[y]=1;
				if(!match[y]||dfs(match[y]))
				{
					match[y]=x;
					return 1;
				}
			}
			else
				delta=min(delta,la[x]+lb[y]-mp[x][y]);
		}
	}
	return 0;
}

int km()
{
	for(int i=1;i<=n;++i)
	{
		match[i]=0;
		la[i]=-inf;
		lb[i]=0;
		for(int j=1;j<=n;++j)
		{
			la[i]=max(la[i],mp[i][j]);
		}
	}
	for(int i=1;i<=n;++i)
	{
		while(1)
		{
			memset(va,0,sizeof(va));
			memset(vb,0,sizeof(vb));
			delta=inf;
			if(dfs(i)) break;
			for(int j=1;j<=n;++j)
			{
				if(va[j]) la[j]-=delta;
				if(vb[j]) lb[j]+=delta;
			}
		}
	}
	int ans=0;
	for(int i=1;i<=n;++i)
		ans+=mp[match[i]][i];
	return ans;
}

int main()
{
	while(~scanf("%d",&n))
	{
		memset(mp,-0x3f,sizeof(mp));
		for(int i=1;i<=n;++i)
		{
			for(int j=1;j<=n;++j)
			{
				scanf("%d",&mp[i][j]);
			}
		}
		printf("%d\n",km());
	}
	return 0;
}
\end{lstlisting}