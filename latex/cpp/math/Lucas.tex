
$$C_N^m \equiv C_{N\ mod\ p}^{m\ mod\ p} * C_{N/p}^{m/p}(\ mod\ p)$$

$$\tbinom N m \equiv \tbinom{N\mod p}{m\mod p} \cdot \tbinom{N/p}{m/p}(\mod p)$$

可理解为将 $N$ 和 $m$ 表示为 $p$ 进制数(形如 $\Sigma N_ip^i$ ),对每一位的 $N_i$ 和 $m_i$ 分别求组合数,再累乘,注意此处的**$p$ 必须为质数**

\begin{lstlisting}
//洛谷P3807 
#include <bits/stdc++.h>
using namespace std;
typedef long long ll;
const int MN = 5e6 + 5;

inline ll qpow(ll a,ll b,int P){	//a^b%P 
	ll ans=1;
	for(;b;b>>=1,a=a*a%P)
		if(b&1) ans=ans*a%P;
return ans;}

ll fct[MN],fi[MN];	//阶乘及其逆元
inline void init(int k,int P){	//打表模P的[1,k]阶乘及其逆元
	fct[0]=1;
	for(int i=1; i<=k; ++i) fct[i]=fct[i-1]*i%P;
	if(k<P){
		fi[k]=qpow(fct[k],P-2,P);
		for(int i=k; i>=1; --i) fi[i-1]=fi[i]*i%P;
	}else{	//k阶乘为0,会把所有逆元都变成0,应从P-1开始
		fi[P-1]=qpow(fct[P-1],P-2,P);
		for(int i=P-1; i>=1; --i) fi[i-1]=fi[i]*i%P;
	}
}

inline int C(int N,int m,int P){	//C_N^m % P
	if(m>N) return 0;
	return fct[N]*fi[m]%P*fi[N-m]%P;
}

//ll lucas(int N,int m,int P){	//递归求C_N^m % P
//	if(!m) return 1;
//	return C(N%P,m%P,P)*lucas(N/P,m/P,P)%P;
//}

int lucas(int N,int m,int P){	//循环求C_N^m % P
	ll rt=1;
	while(N&&m)
		(rt*=C(N%P,m%P,P))%=P,
		N/=P, m/=P;
	return rt;
}

void solve(){
	int n,m,p; scanf("%d%d%d",&n,&m,&p);
	init(n+m,p);
	printf("%lld\n",lucas(n+m,m,p));
}

int main(int argc, char** argv){ 
	int _; scanf("%d",&_); while(_--)
		solve();
	return 0;
}
\end{lstlisting}