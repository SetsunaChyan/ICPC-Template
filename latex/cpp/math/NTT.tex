\begin{lstlisting}
/* 
大致原理:令P=998244353,g=3,g的n次幂模P可得到P种值,称g为模P的原根
O(NlogN)地将一个N次多项式系数表达式转化成关于模P的原根的g的次幂的N个点值 
对两个多项式的点值做乘法,得到新的点值表达式,再O(NlogN)地转回系数表达式
利用R数组和“蝴蝶定理”可以将递归NTT过程转为循环过程,优化复杂度的常数
写得很乱,欢迎有空的同学来更新个好看点的
顺带一提求逆和分治里用的是细节不同的另外两种实现的NTT,不过也都很乱
*/
//洛谷P3803裸的多项式乘积,在大数据下的表现可能反而不如FFT
#include <bits/stdc++.h>
using namespace std;
typedef long long ll;
 
const int MN = 3e6 + 5;
const int P = 998244353;

int qpow(ll a,int b){
	ll ans=1;
	for(;b;b>>=1){
		if(b&1)
			ans=ans*a%P;
		a=a*a%P;}
return ans;}

int read(){
	int k=0, f=1;
	char c=getchar();
	while(!isdigit(c)){ if(c=='-') f=-1; c=getchar(); }
	while(isdigit(c)) k= k*10 + c-48 , c=getchar();
	return k*f;
}
void write(int x){
	if(x<0){ putchar('-'); x=~(x-1); }
	int s[20],top=0;
	while(x){ s[++top]=x%10; x/=10; }
	if(!top) s[++top]=0;
	while(top) putchar(s[top--]+'0');
}

int a[MN],b[MN],c[MN];
int N,M,n,ln;	//左多项式次数,右多项式次数,fft次数及其位数
int R[MN];	//01串逆转后对应的下标
inline void calR(){	//通过NM初始化以上数据
	n=1;
	while(n<=N+M) n<<=1, ++ln;
	for(int i=1; i<n; ++i) R[i] = (R[i>>1] >>1) | ((i&1)<< ln-1);
}
inline void init(){
	N=read(); M=read();
	for(int i=0; i<=N; ++i) a[i]=read();
	for(int i=0; i<=M; ++i) b[i]=read();
	calR();
}
void ntt(int c[], int f=1){	//f取-1时是逆变换
	for(int i=0; i<n; ++i) if(i<R[i]) swap(c[i], c[R[i]]);
	for(int j=1; j<n; j<<=1){
		int j2=j<<1;
		int rt=qpow(3,(P-1)/j2);
		if(f==-1) rt=qpow(rt, P-2);	//逆变换时除以rt,即乘以rt逆元
		for(int k=0; k<n; k+=j2){
			int t=1;
			for(int l=0; l<j; ++l){
				int tl=c[k+l], tr=ll(t)*c[j+k+l]%P;
				c[k+l]=(tl+tr)%P;
				c[j+k+l]=((tl-tr)%P+P)%P;
				t=ll(t)*rt%P;
			}
		}
	}
}

int main(int argc, char** argv) {
//	ios::sync_with_stdio(0);
	init();
	ntt(a), ntt(b);
	for(int i=0; i<=n; ++i) c[i]=ll(a[i])*b[i]%P;
	ntt(c,-1);
	ll mi=qpow(n,P-2);
	for(int i=0; i<=N+M; ++i) write((c[i]*mi)%P), putchar(' ');
	return 0;
}
\end{lstlisting}