\begin{lstlisting}
//大致原理是cdq分治套NTT求f_i = f和g的前i项卷积
//洛谷P4721求f_i = \sigma_{j=1]^i f_{i-j} * g_j 
#include <bits/stdc++.h>
using namespace std;
typedef long long ll;

const int P = 998244353;
const int G = 3;
const int MN = 2e5 + 5;

inline int read(){
	int k=0, f=1;
	char c=getchar();
	while(!isdigit(c)){ if(c=='-') f=-1; c=getchar(); }
	while(isdigit(c)) k=k*10+c-48, c=getchar();
	return k*f;}

int qpow(ll a, int b){
	ll ans=1ll;
	for(;b;b>>=1){
		if(b&1)
			ans=ans*a%P;
		a=a*a%P;}
	return ans;}

inline int inv(ll value){ return qpow(value, P-2); }

namespace NTT{
	int R[MN];	//01串逆转后对应的下标
	void calcR(int len){	//计算len次多项式的R数组
		int loglen = R[0] = 0;
		while(1<<loglen < len) ++loglen;
		for(int i=1; i<len; ++i) R[i] = (R[i>>1] >>1) | ((i&1)<< loglen-1);
	}
	void ntt(int *a, int n, int f=1){	//f取-1时是逆变换
		for(int i=0; i<n; ++i) if(R[i] < i) swap(a[i],a[R[i]]);
		int baseW = qpow(G, (P-1) / n);
		if(f==-1) baseW = inv(baseW);
		for(int len = 2; len <= n; len <<= 1){
			int mid = len >>1;
			int wn = qpow(baseW, n / len);
			for(int *pos = a; pos != a+n; pos += len){
				int w = 1;
				for(int i=0; i<mid; ++i) {
					int x = pos[i], y = (ll)pos[mid + i] * w % P;
					pos[i] = ((ll)x + y) % P;
					pos[mid + i] = ((ll)x - y + P) % P;
					w = (ll)w * wn % P;
				}
			}
		}
	}
	void mult(int *a, int *b, int len){	//封装好的len次多项式乘法,答案存进a
		calcR(len);
		ntt(a,len);
		ntt(b,len);
		for(int i=0; i<len; ++i)  a[i] = (ll)a[i] * b[i] % P;
		ntt(a,len,-1);
		int x = inv(len);
		for(int i=0; i<len; ++i)  a[i] = (ll)a[i] * x % P;
	}
}
using namespace NTT;

int f[MN],g[MN],A[MN],B[MN];	//输出输入及其副本
int n;	//多项式长度

void cdq(int l, int r){	//分治[l,r]
	if(l==r) return;
	int mid = l+r >> 1, len = 1;
	while(len < r-l) len <<= 1;
	cdq(l,mid);
	for(int i=0; i<len; ++i) A[i] = 0, B[i] = 0;
	for(int i=0; i<=mid-l; ++i) A[i] = f[l+i];	//l开始的当前区间的左半f
	for(int i=0; i<r-l; ++i)  B[i] = g[i+1];	//1开始的当前区间长度的g
	mult(A,B,len);	//卷积求出左半端的贡献
	for(int i=mid+1; i<=r; ++i) f[i] = (f[i] + A[i-l-1]) % P;
	cdq(mid+1,r);
}

int main(){
	n = read();
	for(int i=1; i<n; ++i) g[i] = read();
	f[0] = 1;	//题目给的初值
	cdq(0,n-1);
	for(int i=0; i<n; ++i) printf("%d ",f[i]);
	return 0;
}
\end{lstlisting}