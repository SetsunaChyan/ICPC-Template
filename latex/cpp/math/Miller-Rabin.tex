$$二次探测:0<x<p,p是质数 x^2 \equiv 1(\mod p) \Leftrightarrow x=1或x=p-1$$
$$费马探测:0<a<p,p是质数 \Rightarrow a^{p-1} \equiv 1(\mod p)$$
判错概率不超过 $4^{-S}(S为测试次数)$,时间复杂度 $O(S*logn)$
\begin{lstlisting}
//HDU2138多组输入判断n个数中几个是质数
#include <cstdio>
#include <cstdlib>
#include <ctime>
#include <random>
using namespace std;
typedef long long ll;

inline ll qpow(ll a, ll b, ll P){	//a^b%P
	ll ans=1;
	for(; b; b>>=1,a=a*a%P)
		if(b&1) ans=ans*a%P;
return ans;}

inline bool MR(int tc, ll n){	//用随机测试数据tc测n,返回n是质数?1:0
	ll u=n-1, t=0;
	while(!(u&1)) ++t, u>>=1;	//把n-1表示为u*2^t,循环到u为奇数为止
	ll x=qpow(tc,u,n);
	if(x==1) return 1;
	for(int i=1; i<=t; ++i,x=x*x%n)
		if(x!=n-1&&x!=1&&x*x%n==1) return 0;
	return x==1;
}

mt19937_64 rnd(time(0));	//c++11的std的魔法,下文可用rnd()生成[0,2^64)

inline bool isPrime(ll n, int S=10){	//对n测试S次,返回n是质数?1:0
	if(n==2) return 1;
	if(n<2 || !(n&1)) return 0;
//	srand(time(0));
	while(S--)
//		if(!MR((ll)rand()*rand()*rand()%(n-1)+1,n)) return 0;
		if(!MR(rnd()%(n-1)+1,n)) return 0;
	return 1;
}
 
inline void solve(int N){
	int cnt=0;
	while(N--){
		ll n; scanf("%lld",&n);
		if(isPrime(n)) ++cnt;
	}
	printf("%d\n",cnt);
}

int main(int argc, char** argv){
	int _; while(~scanf("%d",&_))
		solve(_);
	return 0;
}
\end{lstlisting}
