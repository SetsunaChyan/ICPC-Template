求解 $x \mod m_i = a_i$ 方程组,其中 $m_i$ 不一定为质数
和CRT本身没啥关系,是用数学归纳法求解齐次同余方程组的
\begin{lstlisting}
//洛谷P4777 
#include <bits/stdc++.h>
typedef long long ll;
const int MN = 3e5 + 5;

ll a[MN],m[MN];

inline ll exgcd(ll a, ll b, ll &x, ll &y){
	if(!b) return x=1, y=0, a;
	ll g=exgcd(b, a%b, x, y);
	ll z=x; x=y; y=z-a/b*y;
return g;}

ll smul(ll a,ll b,ll p){	//记得传参时先给ab取余一发p
	ll ans=0;
	for(;b;b>>=1){
		if(b&1) ans= (ans+a)%p;
		a= (a<<1)%p;}
return ans;}

//ll qmul(ll a,ll b,ll p){	//玄学高精度乘法,备用,可能可以代替上一个
//	a%=p, b%=p;
//	ll t=(long double)a*b/p;
//	ll ans=a*b-t*p;
//return ans<0? ans+p: ans;}

ll excrt(int n){	//解[0,n)
	ll X, Y, M=m[0], ans=a[0];
	for(int i=1; i<n; ++i){
		ll A=M, B=m[i];
		ll c=(a[i]-ans%B+B)%B;	//新同余方程的右部
		ll g=exgcd(A,B,X,Y);
		if(c%g!=0) return -1;
		X=smul(X,c/g,B/g);
		ans+=X*M;
		M*=B/g;
		ans=(ans%M+M)%M;}
return (ans%M+M)%M;}

int main(){
	int n; scanf("%d",&n);
	for(int i=0; i<n; ++i) scanf("%lld%lld",m+i,a+i);
	printf("%lld",excrt(n));
	return 0;
}

\end{lstlisting}