\begin{lstlisting}
/*
求解x^2 % P = n,当P为奇质数时,可用类似虚数平方的方法求解,
类似实数域一元二次方程,答案可能为两不同解,两相同解或无实数解,详见solve()
*/
//洛谷P5491求二次剩余
#include <bits/stdc++.h>
using namespace std;
typedef long long ll;

int P;	//模数,为了避免传参而放在全局变量

inline ll qpow(ll a,int b){	//a^b%P
	ll ans=1;
	for(; b; b>>=1,a=a*a%P)
		if(b&1) ans=ans*a%P;
return ans;}

ll t,tt;	//tt为t的平方。注意!!!复数类中每次乘法都要用到tt!!!
struct CP{	//求解二次剩余专用的魔改复数类 
	ll x,y;
	CP(ll x=0,ll y=0):x(x),y(y){}
	CP operator*=(const CP&r){	//乘以r,模数P为全局变量
		return *this = CP((x*r.x%P+y*r.y%P*tt%P)%P,(y*r.x%P+x*r.y%P)%P);
	}
	CP qpow(int n){	//n次幂,模数P为全局变量
		CP rt = CP(1,0);
		for(; n; n>>=1){
			if(n&1) rt *= *this;
			*this *= *this;
		}
		return *this = rt;
	}
};

int cipolla(int n){	//求x*x%P=n的一个解,P是奇质数,无解时return-1
	if(n==0) return 0; 
	if(qpow(n,(P-1)>>1)==P-1) return -1;	//无解
	srand(time(0));	//初始化随机数种子
	for(;;){	//随机找到一个满足break条件的t即可 
		t=rand()%P;
		tt=(t*t%P-n+P)%P;
		if(qpow(tt,(P-1)>>1)==P-1) break;
	}
	CP rt=CP(t,1).qpow((P+1)>>1);
	return rt.x;
}

inline void solve(){
	int n; scanf("%d%d",&n,&P);	//注意P是全局变量,之后就不传参了 
	int ans=cipolla(n);
	int ans2=(P-ans)%P;	//此处取模纯粹是为了避免ans为0时ans2为P
	if(ans==-1) return puts("Hola!"), void(); 
	if(ans2<ans) swap(ans,ans2);
	if(ans!=ans2) printf("%d %d\n",ans,ans2);
	else printf("%d\n",ans);
}

int main(int argc, char** argv){
	int _; scanf("%d",&_); while(_--)
		solve();
	return 0;
}
\end{lstlisting}
