费马小定理: $p$ 为质数时,$$a^{p-1} \equiv 1 (\mod p) $$
欧拉定理: \textbf{$a,p$ 互质时}, $$a^{\phi(p)} \equiv 1 (\mod p) $$ 
扩展欧拉定理:  \textbf{$a,p$ 不互质时}, 
$$
a^b\equiv
\begin{cases}
a^{b \mod \phi(p) + \phi(p)} \ \ {b \ge \phi(p)} \\
a^b \ \ {0 \le b < \phi(p)}
\end{cases}
(\mod p)
$$
\begin{lstlisting}
//SP10050:用扩展欧拉定理求乘方塔a^a^a……(b个a)的后九位数
#include <bits/stdc++.h>
using namespace std;
typedef long long ll;

struct ST{
	int v;
	bool ge;	//大于等于模数与否
	ST(int v=0,bool g=0): v(v), ge(g) {} 
};

ST qpow(ll a,ll b,int p){	//快速幂过程中取模过p iff 返回值.ge==1
	ll ans=1ll;
	bool ge=0;
	while(b){
		if(b&1)
			ans*=a,
			ge |= ans>=p,
			ans%=p;
		b>>=1;
		if(!b) break;	//防止被没有乘到的a更新ge
		a*=a;
		ge |= a>=p,	//注意ans*取余后的a可能更新不了ge,在这也要更新 
		a%=p;
	}
	return ST(ans,ge);
}

map<int,int>eu;
ll euler(ll n){	//欧拉函数值
	if(eu[n]) return eu[n];
	ll n0=n, ans=n, ed=sqrt(n);
	for(int i=2; i<=ed; ++i)
		if(n%i==0){
			ans-=ans/i;
			while(n%i==0) n/=i;
		}
	if(n>1) ans-=ans/n;
	return eu[n0]=ans;
}

ST tower(ll a,ll b,int p){	//计算b层a取余p的值 
	if(p==1) return ST(0,1);	//特判取模1的特殊情况 
	if(a==1) return ST(1,0);	//特判不取模1但底为1的特殊情况
	if(b==1) return a<p? ST(a,0): ST(a%p,1);	//递归终点
	int phip=euler(p);
	ST ans=tower(a,b-1,phip);	//递归计算取余phip后的指数
	if(ans.ge) ans.v+=phip;	//扩展欧拉定理 
	return qpow(a, ans.v, p);
}

void solve(){
	ll a,b; scanf("%lld%lld",&a,&b);
	if(b==0) return printf("%d\n",1), void(0);
	else if(a==0) return printf("%d\n", b%2? 0: 1), void(0);
	ST ans=tower(a,b,1000000000);
	if(ans.ge) printf("...%09d\n",ans.v);
	else printf("%d\n",ans.v);
}

int main(int argc, char** argv) {
	int _; scanf("%d",&_); while(_--)
		solve();
	return 0;
}

\end{lstlisting}