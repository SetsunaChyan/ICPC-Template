$Pollard$ 设计的求大数因子方法,因为随机找的数会形成环形,因此常称为 $\rho$ 算法
大致原理是随机找几个数求差,让差与大数 $N$ 求 $gcd$ ,据说复杂度约 $O(N^{1/4})$
随机找数的方法见 $rho$ 函数。若要求质因子,一般需要用一下 $Miller Rabin$ 判素数
\begin{lstlisting}
//洛谷P4718求最大质因子
#include <cstdio>
#include <cstdlib>
#include <ctime>
#include <random>
#include <algorithm>
using namespace std;
typedef long long ll;
typedef unsigned long long ull;
typedef long double lb;

ll m, ans;	//对每个m,将其最大质因子存进ans

mt19937_64 rnd(time(0));	//c++11的std的魔法,下文可用rnd()生成[0,2^64)

ll read(){
	ll k=0; int f=1;
	char c=getchar();
	while(!isdigit(c)){ if(c=='-') f=-1; c=getchar(); }
	while(isdigit(c)) k= k*10 + c-48, c=getchar();
	return k*f;
}

inline ll Abs(ll x){ return x<0 ? -x : x;}	//取绝对值

inline ll qmul(ull x, ull y, ll p){	//O(1)x*y%p
	return (x*y - (ull)((lb)x/p*y)*p + p)%p;
}

inline ll qpow(ll x, ll y, ll p){	//x^y%p
	ll res=1;
	for(; y; y>>=1, x=qmul(x,x,p))
		if(y&1) res=qmul(res,x,p);
	return res;
}

inline bool MR(int tc, ll p){	//miller rabin判质数,用tc检测p是不是质数
	if(qpow(tc,p-1,p)!=1) return 0;	//费马小定理
	ll y=p-1, z;
	while(!(y&1)){	//二次探测
		y>>=1;
		z=qpow(tc,y,p);
		if(z!=1 && z!=p-1) return 0;
		if(z==p-1) return 1;
	}
	return 1;
}

inline bool isPrime(ll x){	//用5个小质数做MR测试,返回n是质数?1:0
	if(x<2) return 0;
    if(x==2 || x==3 || x==5 || x==7 || x==43) return 1;
    return MR(2,x) && MR(3,x) && MR(5,x) && MR(7,x) && MR(43,x);
}

inline ll rho(ll p){	//玄学求出p的非平凡因子
	ll x,y,s,c;	//s用来存(y-x)的乘积
	for(;;){	//求出一个因子来
		y=x=rnd()%p;
		c=rnd()%p;
		s=1;
		int i=0, I=1;
		while(++i){	//开始玄学倍增生成
			x=(qmul(x,x,p)+c)%p;	//以平方再+c的方式生成下一个x
			s=qmul(s,Abs(y-x),p);	//将每一次的(y-x)都累乘起来
			if(x==y || !s) break;	//换下一组,当s=0时,继续下去是没意义的
			if(!(i%127) || i==I){	//每127次求一次gcd,以及按j倍增的求gcd
				ll g=__gcd(s,p);
				if(g>1)	return g;
				if(i==I) y=x, I<<=1;	//维护倍增正确性,并判环
			}
		}
	}
}

inline void PR(ll p){	//找p的最大质因子,存进全局变量ans
	if(p<=ans) return ;	//最优性剪枝
	if(isPrime(p)) return ans=p, void();
	ll pi=rho(p);	//玄学求出任一因子p_i
	while(p%pi==0) p/=pi;
	return PR(pi), PR(p);	//向下分治
}

int main(){
    int n=read();
    for(int i=0; i<n; ++i){
        ans=1;
		PR( m=read() );
        if(ans==m) puts("Prime");
        else printf("%lld\n",ans);
    }
    return 0;
}
\end{lstlisting}

