\begin{lstlisting}
/* 
大致原理:O(NlogN)地将一个N次多项式系数表达式转化成关于(1的N次复数根)的N个点值
对两个多项式的点值做乘法,得到新的点值表达式,再O(NlogN)地转回系数表达式
利用R数组和“蝴蝶定理”可以将递归FFT过程转为循环过程,优化复杂度的常数
写得很乱,欢迎有空的同学来更新个好看点的
*/
//洛谷P3803裸的多项式乘积,利用了(a+bi)^2 = (a^2-b^2) + 2abi减少变换次数
#include <bits/stdc++.h>
using namespace std;
typedef long long ll;
typedef double db;
 
const int MN = 3e6 + 5;
const db pi = acos(-1);

struct CP{
	db x,y;	//实部,虚部
	CP (db x=0, db y=0): x(x), y(y) {}
	CP operator+(CP &t){ return CP(x+t.x, y+t.y); }
	CP operator-(CP &t){ return CP(x-t.x, y-t.y); }
	CP operator*(CP &t){ return CP(x*t.x - y*t.y, x*t.y + y*t.x); }
}a[MN];

int read(){
	int k=0, f=1;
	char c=getchar();
	while(!isdigit(c)){ if(c=='-') f=-1; c=getchar(); }
	while(isdigit(c)) k= k*10 + c-48 , c=getchar();
	return k*f;
}
void write(int x){
	if(x<0){ putchar('-'); x=~(x-1); }
	int s[20],top=0;
	while(x){ s[++top]=x%10; x/=10; }
	if(!top) s[++top]=0;
	while(top) putchar(s[top--]+'0');
}

int N,M,n,ln;	//左多项式次数,右多项式次数,fft次数及其位数
int R[MN];	//01串逆转后对应的下标
inline void calR(){	//通过NM初始化以上数据
	n=1;
	while(n<=N+M) n<<=1, ++ln;
	for(int i=1; i<n; ++i) R[i] = (R[i>>1] >>1) | ((i&1)<< ln-1);
}
inline void init(){
	N=read(); M=read();
	for(int i=0; i<=N; ++i) a[i].x=read();
	for(int i=0; i<=M; ++i) a[i].y=read();
	calR();
}
void fft(CP c[], int f=1){	//f取-1时是逆变换
	for(int i=0; i<n; ++i) if(i<R[i]) swap(c[i], c[R[i]]);
	for(int j=1; j<n; j<<=1){
		CP wn(cos(pi/j), f*sin(pi/j));
		for(int k=0; k<n; k+=(j<<1)){
			CP t(1, 0);
			for(int l=0; l<j; ++l){
				CP cl=c[k+l], cr=t*c[j+k+l];
				c[k+l]=cl+cr;
				c[j+k+l]=cl-cr;
				t=t*wn;
			}
		}
	}
}

int main(int argc, char** argv) {
//	ios::sync_with_stdio(0);
	init();
	fft(a);//, fft(b);
	for(int i=0; i<=n; ++i) a[i]=a[i]*a[i];
	fft(a,-1);
	for(int i=0; i<=N+M; ++i) write(int(a[i].y/n/2 + 0.5)), putchar(' ');
	return 0;
}
\end{lstlisting}