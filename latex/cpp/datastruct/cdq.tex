\begin{lstlisting}
    #include <bits/stdc++.h>
using namespace std;

/*
    解决有等于的三维偏序
	严格小于等于的个数,可以解决重复问题,有离散化
*/

const int maxn=500005;

int n,k;
int cnt[maxn];//save the ans
struct ss{
	int a,b,c,w,ans;
}tmps[maxn],s[maxn];//struct
bool cmp1(ss x,ss y){//sort1
	if(x.a==y.a){
		if(x.b!=y.b) return x.b<y.b;
		else return x.c<y.c;
	}
	else return x.a<y.a;
}
bool cmp2(ss x,ss y){//sort2
	if(x.b!=y.b) return x.b<y.b;
	else return x.c<y.c;
}

struct tree_array{//tree_array
	int tr[maxn+5],n;
	int lowbit(int x){return x&-x;}
	int ask(int x){int ans=0;for(;x;x-=lowbit(x))ans+=tr[x];return ans;}
	void add(int x,int y){for(;x<=n;x+=lowbit(x))tr[x]+=y;}	
}t;

void cdq(int l,int r){
	if(l==r) return;
	int m=l+r>>1;
	cdq(l,m);
	cdq(m+1,r);
	sort(s+l,s+m+1,cmp2);
	sort(s+m+1,s+r+1,cmp2);//sort2
	int i=l,j=m+1;
	for(;j<=r;++j){
		while(i<=m&&s[i].b<=s[j].b){//the second dimension
			t.add(s[i].c,s[i].w);//use the tree_array to save the ans
			++i;
		}
		s[j].ans+=t.ask(s[j].c);//contribution
	}
	for(int j=l;j<i;++j)
		t.add(s[j].c,-s[j].w);//init the first half
}

int main(){
	scanf("%d%d",&n,&k);
	for(int i=1;i<=n;++i)
		scanf("%d%d%d",&tmps[i].a,&tmps[i].b,&tmps[i].c);
	sort(tmps+1,tmps+1+n,cmp1);//sort1
	int now=0,nn=0;
	for(int i=1;i<=n;++i){
		now++;
		if(tmps[i].a!=tmps[i+1].a||tmps[i].b!=tmps[i+1].b
		||tmps[i].c!=tmps[i+1].c){
			s[++nn]=tmps[i];
			s[nn].w=now;
			now=0;
		}//compress the same
	}
	t.n=maxn;//tree_array on the range
	cdq(1,nn);
	for(int i=1;i<=nn;++i)
		cnt[s[i].ans+s[i].w-1]+=s[i].w;//
	for(int i=0;i<n;++i)
		printf("%d\n",cnt[i]);
	return 0;
}
\end{lstlisting}