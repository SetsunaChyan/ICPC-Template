\begin{lstlisting}
#include <bits/stdc++.h>
#define x1 xa
#define x2 xb
#define y1 ya
#define y1 yb
using namespace std;
typedef long long ll;
const int N=500005;

//luogu P5490

/*
上面的define是为了防止某些时候的CE
时刻记得节点[3,3]表示的是3号Y坐标与4号Y坐标构成的线段长度
*/

int n;
ll x1,y1,x2,y2;
vector<ll> Y;

struct line
{
	ll x,y1,y2;
	int mark;
};
vector<line> v;
bool cmp(line a,line b){return a.x<b.x;}

struct seg
{
	int l,r,cov;
	ll len;
}tr[N<<2];

void build(int p,int l,int r)
{
	tr[p].l=l,tr[p].r=r;
	tr[p].len=tr[p].cov=0;
	if(l==r) return;
	int m=l+r>>1;
	build(p<<1,l,m);
	build(p<<1|1,m+1,r);
}

void up(int p)
{
	int l=tr[p].l,r=tr[p].r;
	if(tr[p].cov>0)
		tr[p].len=Y[r]-Y[l-1];//
	else
		tr[p].len=tr[p<<1].len+tr[p<<1|1].len;
}

void modi(int L,int R,int V,int p=1)
{
	int l=tr[p].l,r=tr[p].r;
	if(L<=l&&r<=R)
	{
		tr[p].cov+=V;
		up(p);
		return;
	}
	int m=l+r>>1;
	if(L<=m) modi(L,R,V,p<<1);
	if(R>m) modi(L,R,V,p<<1|1);
	up(p);
}

int main()
{
	scanf("%d",&n);
	for(int i=1;i<=n;++i)
	{
		scanf("%lld%lld%lld%lld",&x1,&y1,&x2,&y2);
		Y.push_back(y1);
		Y.push_back(y2);
		v.push_back((line){x1,y1,y2,1});
		v.push_back((line){x2,y1,y2,-1});
	}
	sort(v.begin(),v.end(),cmp);
	sort(Y.begin(),Y.end());
	for(auto &i:v)
	{
		i.y1=lower_bound(Y.begin(),Y.end(),i.y1)-Y.begin()+1;
		i.y2=lower_bound(Y.begin(),Y.end(),i.y2)-Y.begin()+1;
	}
	build(1,1,Y.size());
	ll ans=0;
	for(int i=0;i<=v.size()-2;++i)
	{
		modi(v[i].y1,v[i].y2-1,v[i].mark);//
		ans+=tr[1].len*(v[i+1].x-v[i].x);
	}
	printf("%lld",ans);
	return 0;
}
\end{lstlisting}