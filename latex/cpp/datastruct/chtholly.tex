\begin{lstlisting}
#include <bits/stdc++.h>
#define int long long //be careful
//CF896C
using namespace std;

/*
	珂朵莉树的左右split顺序很重要,并且set集合一开始不要为空,否则会RE
*/

const int N=1000005;

int qpow(int a,int b,int mod)
{
	int res=1,tmp=a%mod;
	while(b)
	{
		if(b&1) res=res*tmp%mod;
		tmp=tmp*tmp%mod;
		b>>=1;
	}
	return res;
}

struct node
{
	int l,r;
	mutable int v;
	node(int L,int R=-1,int V=0):l(L),r(R),v(V){}
	bool operator < (const node& o)const{return l<o.l;}
};
set<node> s;
typedef set<node>::iterator it;

it split(int pos)
{
	it i=s.lower_bound(node(pos));
	if(i!=s.end()&&i->l==pos) return i;
	--i;
	int L=i->l,R=i->r,V=i->v;
	s.erase(i);
	s.insert(node(L,pos-1,V));
	return s.insert(node(pos,R,V)).first;
}

void assign(int l,int r,int val)
{
	it ir=split(r+1),il=split(l);
	s.erase(il,ir);
	s.insert(node(l,r,val));
}

void add(int l,int r,int val)
{
	it ir=split(r+1),il=split(l);
	for(;il!=ir;il++)
		il->v+=val;
}

int rk(int l,int r,int k)
{
	vector<pair<int,int>> v;
	it ir=split(r+1),il=split(l);
	for(;il!=ir;il++)
		v.emplace_back(il->v,il->r-il->l+1);
	sort(v.begin(),v.end());
	for(int i=0;i<v.size();++i)
	{
		k-=v[i].second;
		if(k<=0) return v[i].first;
	}
	return -1; //can't find
}

int sum(int l,int r,int ex,int mod)
{
	it ir=split(r+1),il=split(l);
	int res=0;
	for(;il!=ir;il++)
		res=(res+qpow(il->v,ex,mod)*(il->r-il->l+1)%mod)%mod;
	return res;
}

inline int read(){
    char ch=getchar();int s=0,w=1;
    while(ch<48||ch>57){if(ch=='-')w=-1;ch=getchar();}
    while(ch>=48&&ch<=57){s=(s<<1)+(s<<3)+ch-48;ch=getchar();}
    return s*w;
}
inline void write(int x){
    if(x<0)putchar('-'),x=-x;
    if(x>9)write(x/10);
    putchar(x%10+48);
}
//Fast I/O

int n,m,seed,vmax,a[N];
int rnd()
{
	int ret=seed;
	seed=(seed*7+13)%1000000007;
	return ret;
}

signed main()
{
	n=read(),m=read(),seed=read(),vmax=read();
	for(int i=1;i<=n;++i)
	{
		a[i]=(rnd()%vmax)+1;
		s.insert(node(i,i,a[i]));
	}
	for(int i=1;i<=m;++i)
	{
		int op=(rnd()%4)+1;
		int l=(rnd()%n)+1;
		int r=(rnd()%n)+1;
		if(l>r) swap(l,r);
		int x,y;
		if(op==3) x=(rnd()%(r-l+1))+1;
		else x=(rnd()%vmax)+1;
		if(op==4) y=(rnd()%vmax)+1;
		switch(op)
		{
			case 1:
				add(l,r,x);break;
			case 2:
				assign(l,r,x);break;
			case 3:
				write(rk(l,r,x)),puts("");break;
			case 4:
				write(sum(l,r,x,y)),puts("");break;
		}
			
	}
	return 0;
}
\end{lstlisting}