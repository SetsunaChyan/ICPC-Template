\begin{lstlisting}
#include <bits/stdc++.h>
//CF 786B

/*
	建立两棵线段树分别表示出区间和入区间
	第1到n标号的节点表示原来的n个点
*/

using namespace std;
const int N=1e6+5;//original N=1e5
typedef long long ll;
const ll inf=0x3f3f3f3f3f3f3f3f;
typedef pair<ll,int> pli;

int n,q,s;
ll ans[N];
vector<int> e[N];
vector<ll> val[N];
bool vis[N];

void add(int x,int y,int z)
{
	e[x].push_back(y);
	val[x].push_back(z);
}

int cnt;
struct seg
{
	int id[N];
	void up(int p,bool flg)
	{
		int u=id[p];
		int v=id[p<<1];
		if(flg) swap(u,v);
		add(u,v,0ll);
		u=id[p];
		v=id[p<<1|1];
		if(flg) swap(u,v);
		add(u,v,0ll);
	}
	void build(int flg,int p=1,int l=1,int r=n)
	{
		id[p]=++cnt;
		if(l==r)
		{
			int u=id[p];
			int v=l;
			if(flg) swap(u,v);
			add(u,v,0ll);
			return;
		}
		int m=l+r>>1;
		build(flg,p<<1,l,m);
		build(flg,p<<1|1,m+1,r);
		up(p,flg);
	}
	void segadd(int u,ll w,int L,int R,bool flg,int p=1,int l=1,int r=n)
	{
		if(L<=l&&r<=R)
		{
			int v=id[p];
			if(flg) swap(u,v);
			add(u,v,w);
			return;
		}
		int m=l+r>>1;
		if(L<=m) segadd(u,w,L,R,flg,p<<1,l,m);
		if(R>m) segadd(u,w,L,R,flg,p<<1|1,m+1,r);
	}
}in,out;

void dij()
{
	priority_queue<pli,vector<pli>,greater<pli>> q;
	q.emplace(0ll,s);
	memset(ans,0x3f3f,sizeof(ans));
	ans[s]=0;
	while(!q.empty())
	{
		int t=q.top().second;q.pop();
		if(vis[t]) continue;
		vis[t]=1;
		for(int i=0;i<e[t].size();++i)
		{
			int y=e[t][i];
			ll v=val[t][i];
			if(ans[t]+v<ans[y])
			{
				ans[y]=ans[t]+v;
				q.emplace(ans[y],y);
			}
		}
	}
	for(int i=1;i<=n;++i)
		printf("%lld ",(ans[i]==inf?-1ll:ans[i]));
}

int main()
{
	scanf("%d%d%d",&n,&q,&s);
	cnt=n;
	//the first n points won't change
	in.build(0);
	out.build(1);
	//in tree no flip,out tree must flip
	for(int i=1;i<=q;++i)
	{
		int t,v,u,l,r;
		ll w;
		scanf("%d",&t);
		if(t==1)
		{
			scanf("%d%d%lld",&v,&u,&w);
			l=r=u;
			t=2;
		}
		else scanf("%d%d%d%lld",&v,&l,&r,&w);
		if(t==2) in.segadd(v,w,l,r,0);
		else if(t==3) out.segadd(v,w,l,r,1);
	}
	dij();
	return 0;
}
\end{lstlisting}